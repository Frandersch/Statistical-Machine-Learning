% Options for packages loaded elsewhere
\PassOptionsToPackage{unicode}{hyperref}
\PassOptionsToPackage{hyphens}{url}
%
\documentclass[
]{article}
\usepackage{amsmath,amssymb}
\usepackage{iftex}
\ifPDFTeX
  \usepackage[T1]{fontenc}
  \usepackage[utf8]{inputenc}
  \usepackage{textcomp} % provide euro and other symbols
\else % if luatex or xetex
  \usepackage{unicode-math} % this also loads fontspec
  \defaultfontfeatures{Scale=MatchLowercase}
  \defaultfontfeatures[\rmfamily]{Ligatures=TeX,Scale=1}
\fi
\usepackage{lmodern}
\ifPDFTeX\else
  % xetex/luatex font selection
\fi
% Use upquote if available, for straight quotes in verbatim environments
\IfFileExists{upquote.sty}{\usepackage{upquote}}{}
\IfFileExists{microtype.sty}{% use microtype if available
  \usepackage[]{microtype}
  \UseMicrotypeSet[protrusion]{basicmath} % disable protrusion for tt fonts
}{}
\makeatletter
\@ifundefined{KOMAClassName}{% if non-KOMA class
  \IfFileExists{parskip.sty}{%
    \usepackage{parskip}
  }{% else
    \setlength{\parindent}{0pt}
    \setlength{\parskip}{6pt plus 2pt minus 1pt}}
}{% if KOMA class
  \KOMAoptions{parskip=half}}
\makeatother
\usepackage{xcolor}
\usepackage[margin=1in]{geometry}
\usepackage{graphicx}
\makeatletter
\def\maxwidth{\ifdim\Gin@nat@width>\linewidth\linewidth\else\Gin@nat@width\fi}
\def\maxheight{\ifdim\Gin@nat@height>\textheight\textheight\else\Gin@nat@height\fi}
\makeatother
% Scale images if necessary, so that they will not overflow the page
% margins by default, and it is still possible to overwrite the defaults
% using explicit options in \includegraphics[width, height, ...]{}
\setkeys{Gin}{width=\maxwidth,height=\maxheight,keepaspectratio}
% Set default figure placement to htbp
\makeatletter
\def\fps@figure{htbp}
\makeatother
\setlength{\emergencystretch}{3em} % prevent overfull lines
\providecommand{\tightlist}{%
  \setlength{\itemsep}{0pt}\setlength{\parskip}{0pt}}
\setcounter{secnumdepth}{5}
\usepackage[german]{babel}
\usepackage{mathtools}
\usepackage{tikz}
\usepackage{pgf}
\usepackage{csquotes}
\AtBeginDocument{
\renewcommand{\maketitle}{}
}
\PassOptionsToPackage{a4paper,margin = 2.5cm}{geometry}
\usepackage{geometry}
\newcommand{\bcenter}{\begin{center}}
\newcommand{\ecenter}{\end{center}}
\renewcommand{\contentsname}{Inhalt}
\usepackage{blindtext}
\usepackage[backend=biber, style = apa]{biblatex}
\addbibresource{Literatur.bib}
\ifLuaTeX
  \usepackage{selnolig}  % disable illegal ligatures
\fi
\usepackage[]{biblatex}
\addbibresource{Literatur.bib}
\IfFileExists{bookmark.sty}{\usepackage{bookmark}}{\usepackage{hyperref}}
\IfFileExists{xurl.sty}{\usepackage{xurl}}{} % add URL line breaks if available
\urlstyle{same}
\hypersetup{
  pdftitle={Ergebnisse},
  pdfauthor={Franz Andersch \& Niklas Münz},
  hidelinks,
  pdfcreator={LaTeX via pandoc}}

\title{Ergebnisse}
\author{Franz Andersch \& Niklas Münz}
\date{2024-08-22}

\begin{document}
\maketitle

\subsection{Analyse}

Bevor die Ergebnisse erläutert werden, wird kurz auf die Vorgehensweise
bei der Analyse eingegangen. In die Analyse werden fünf Modelle
einbezogen: SVM mit linearem, polynomialem und radialem Kern, sowie
regularisierte logistische Regression und k-nearest neighbours.\newline
Vor dem erstellen der Modelle wird ein Tuning der Hyperparameter je
Modell durchgeführt. Dafür wird die Bayesian Optimization Methode
genutzt, welche ein iterativer Algorithmus ist. Hierbei werden die
nächsten Evaluierungspunkte basierend auf zuvor beobachteten Ergebnissen
bestimmt \parencite{yangHyperparameterOptimizationMachine2020}. Der
Algorithmus basiert auf zwei Hauptkomponenten: einem Surrogatmodell und
einer Akqusitionsfunktion. Das Surrogatmodell, wofür hier ein Gaussian
Process genutzt wird, passt die bisher beobachteten Punkte an die
Zielfunktion an. Die Akquisitionsfunktion wählt dann die nächsten Punkte
aus, wobei ein Gleichgewicht zwischen der Erkundung neuer Bereiche und
der Nutzung vielversprechender Regionen angestrebt wird. Dafür wird hier
der Ansatz des Upper-Confidence-Bound genutzt, welcher obere
Konfidenzgrenzen nutzt um den Verlust gegenüber der besten möglichen
Entscheidung, während der Optimierung zu minimieren
\parencite{snoekPracticalBayesianOptimization2012}. \begin{align}
a_{UCB}(\mathbf{x};~\left\{\mathbf{x}_n,y_n\right\},\theta) = \mu(\mathbf{x};~\left\{\mathbf{x}_n,y_n\right\},\theta) - k \sigma(\mathbf{x};~\left\{\mathbf{x}_n,y_n\right\},\theta)
\end{align} Die Bayesian Optimization wird genutzt, da sie eine schnelle
Konvergenz für stetige Hyperparameter aufweist
\parencite{yangHyperparameterOptimizationMachine2020}. Als
Evaluierungskriterium wird die Genauigkeit der Modelle, welche durch den
Anteil der korrekt klassifizierten Beobachtungen wiedergegeben wird,
verwendet.

Basierend auf den Ergebnissen des Tuning werden die oben genannten
Modelle erstellt. Daraufhin werden die Genauigkeit, die Receiver
Operating Characteristic Kurve (ROC-Kurve) bzw. der Area Under The Curve
Wert (AUC-Wert) und der F1-Score für jedes Modell bestimmt.\newline Die
ROC-Kurve ist eine grafische Darstellung der Leistungsfähigkeit eines
Klassifikationsmodells, wobei die Sensitivität auf der y-Achse von 0 bis
1 gegen die Spezifität auf der x-Achse von 1 bis 0 abgetragen wird
\parencite{fawcettIntroductionROCAnalysis2006}. Sensitivität und
Spezifität ergeben sich aus: \begin{align}
Sensitivität=\frac{korrekt~Positiv}{korrekt~Positiv~+~falsch~Negativ}
\end{align} \begin{align}
Spezifität=\frac{korrekt~Negativ}{falsch~Positiv~+~korrekt~Negativ}
\end{align} Positiv ist in diesem Fall gleichbedeutend mit Klasse 1 und
Negativ mit Klasse 2. Die ROC-Kurve zeigt dann den Zusammenhang zwischen
dem Nutzen (korrekt Positive) und den Kosten (falsch Positive) auf. Eine
ideale Kurve läuft nah am linken, oberen Rand der Grafik, da hier
bereits bei sehr hoher Spezifität (hohe Anzahl korrekt Negative) eine
hohe Sensitivität (hohe Anzahl korrekt Positive) erreicht wird. Der
AUC-Wert bezieht sich auf die Fläche unterhalb der Kurve und liegt somit
im Intervall {[}0,1{]}, wobei ein Wert von 1 für eine perfekte
Klassifikation spricht, während ein Wert von 0.5 glechbedeutend mit
einer rein zufälligen Zuordnung der Klassen spricht.\newline Für den
F1-Score ist außerdem die Präzision von Bedeutung, die sich wie folgt
berechnet\parencite{fawcettIntroductionROCAnalysis2006}: \begin{align}
Präzision=\frac{korrekt~Positiv}{korrekt~Positiv~+~falsch~Positiv}
\end{align} Der F1-Score beschreibt das harmonische Mittel zwischen
Präzision und Sensitivität und drückt folglich die Fähigkeit des
Modells, gleichzeitig falsch Positive und falsch Negative zu minimieren,
aus. \begin{align}
F1\text{-}Score=\frac{2}{1/Präzision~+~1/Sensitivität}
\end{align}

\subsection{Auswertung}

\printbibliography

\end{document}
