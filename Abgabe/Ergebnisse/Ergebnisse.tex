% Options for packages loaded elsewhere
\PassOptionsToPackage{unicode}{hyperref}
\PassOptionsToPackage{hyphens}{url}
%
\documentclass[
]{article}
\usepackage{amsmath,amssymb}
\usepackage{iftex}
\ifPDFTeX
  \usepackage[T1]{fontenc}
  \usepackage[utf8]{inputenc}
  \usepackage{textcomp} % provide euro and other symbols
\else % if luatex or xetex
  \usepackage{unicode-math} % this also loads fontspec
  \defaultfontfeatures{Scale=MatchLowercase}
  \defaultfontfeatures[\rmfamily]{Ligatures=TeX,Scale=1}
\fi
\usepackage{lmodern}
\ifPDFTeX\else
  % xetex/luatex font selection
\fi
% Use upquote if available, for straight quotes in verbatim environments
\IfFileExists{upquote.sty}{\usepackage{upquote}}{}
\IfFileExists{microtype.sty}{% use microtype if available
  \usepackage[]{microtype}
  \UseMicrotypeSet[protrusion]{basicmath} % disable protrusion for tt fonts
}{}
\makeatletter
\@ifundefined{KOMAClassName}{% if non-KOMA class
  \IfFileExists{parskip.sty}{%
    \usepackage{parskip}
  }{% else
    \setlength{\parindent}{0pt}
    \setlength{\parskip}{6pt plus 2pt minus 1pt}}
}{% if KOMA class
  \KOMAoptions{parskip=half}}
\makeatother
\usepackage{xcolor}
\usepackage[margin=1in]{geometry}
\usepackage{graphicx}
\makeatletter
\def\maxwidth{\ifdim\Gin@nat@width>\linewidth\linewidth\else\Gin@nat@width\fi}
\def\maxheight{\ifdim\Gin@nat@height>\textheight\textheight\else\Gin@nat@height\fi}
\makeatother
% Scale images if necessary, so that they will not overflow the page
% margins by default, and it is still possible to overwrite the defaults
% using explicit options in \includegraphics[width, height, ...]{}
\setkeys{Gin}{width=\maxwidth,height=\maxheight,keepaspectratio}
% Set default figure placement to htbp
\makeatletter
\def\fps@figure{htbp}
\makeatother
\setlength{\emergencystretch}{3em} % prevent overfull lines
\providecommand{\tightlist}{%
  \setlength{\itemsep}{0pt}\setlength{\parskip}{0pt}}
\setcounter{secnumdepth}{5}
\usepackage[german]{babel}
\usepackage{mathtools}
\usepackage{tikz}
\usepackage{pgf}
\usepackage{csquotes}
\AtBeginDocument{
\renewcommand{\maketitle}{}
}
\PassOptionsToPackage{a4paper,margin = 2.5cm}{geometry}
\usepackage{geometry}
\newcommand{\bcenter}{\begin{center}}
\newcommand{\ecenter}{\end{center}}
\renewcommand{\contentsname}{Inhalt}
\usepackage{blindtext}
\usepackage[backend=biber, style = apa]{biblatex}
\addbibresource{Literatur.bib}
\ifLuaTeX
  \usepackage{selnolig}  % disable illegal ligatures
\fi
\usepackage[]{biblatex}
\addbibresource{Literatur.bib}
\IfFileExists{bookmark.sty}{\usepackage{bookmark}}{\usepackage{hyperref}}
\IfFileExists{xurl.sty}{\usepackage{xurl}}{} % add URL line breaks if available
\urlstyle{same}
\hypersetup{
  pdftitle={Ergebnisse},
  pdfauthor={Franz Andersch \& Niklas Münz},
  hidelinks,
  pdfcreator={LaTeX via pandoc}}

\title{Ergebnisse}
\author{Franz Andersch \& Niklas Münz}
\date{2024-08-20}

\begin{document}
\maketitle

\subsection{Analyse}

Bevor die Ergebnisse dargelegt werden, wird kurz auf die Vorgehensweise
bei der Analyse eingegangen. In die Analyse werden fünf Modelle
einbezogen: SVM mit linearem, polynomialem und radialem Kern, sowie
regularisierte logistische Regression und k-nearest neighbours.\newline
Vor dem erstellen der Modelle wird ein Tuning der Hyperparameter je
Modell durchgeführt. Dafür wird die Bayesian Optimization Methode
genutzt, welche ein iterativer Algorithmus ist. Hierbei werden die
nächsten Evaluierungspunkte basierend auf zuvor beobachteten Ergebnissen
bestimmt \parencite{yangHyperparameterOptimizationMachine2020}. Der
Algorithmus basiert auf zwei Hauptkomponenten: einem Surrogatmodell und
einer Akqusitionsfunktion. Das Surrogatmodell, wofür hier ein Gaussian
Process genutzt wird, passt die bisher beobachteten Punkte an die
Zielfunktion an. Die Akquisitionsfunktion wählt dann die nächsten Punkte
aus, wobei ein Gleichgewicht zwischen der Erkundung neuer Bereiche und
der Nutzung vielversprechender Regionen angestrebt wird. Dafür wird hier
der Ansatz des Upper-Confidence-Bound genutzt, welcher obere
Konfidenzgrenzen nutzt um den Verlust gegenüber der besten möglichen
Entscheidung, während der Optimierung zu minimieren
\parencite{snoekPracticalBayesianOptimization2012}. Die Bayesian
Optimization wird genutzt, da sie eine schnelle Konvergenz für stetige
Hyperparameter aufweist
\parencite{yangHyperparameterOptimizationMachine2020}. Als
Evaluierungskriterium wird die Genauigkeit der Modelle, welche durch den
Anteil der korrekt klassifizierten Beobachtungen wiedergegeben wird,
verwendet.\newline Basierend auf den Ergebnissen des Tuning werden die
oben genannten Modelle erstellt. Daraufhin werden die Genauigkeit und
die Receiver Operating Characteristic Kurve (ROC-Kurve) bzw. der Area
Under The Curve Wert (AUC-Wert) für jedes Modell bestimmt. Die ROC-Kurve
ist eine grafische Darstellung der Leistungsfähigkeit eines
Klassifikationsmodells, wobei die Sensitivität gegen die Spezifität
abgetragen wird \parencite{fawcettIntroductionROCAnalysis2006}. Die
Sensitivität gibt den Anteil der korrekt als positiv (hier
gleichbedeutend mit Klasse 1) klassifizierten Beobachtungen an während
die Spezifität den Anteil der korrekt als negativ (hier gleichbedeutend
mit Klasse 2) klassifizierten Beobachtungen angibt. Der AUC-Wert bezieht
sich auf die Fläche unterhalb die Kurve und liegt somit im Intervall
{[}0,1{]}, wobei ein Wert von 1 für eine perfekte Klassifikation
spricht, während ein Wert von 0.5 für eine rein zufällige Zuordnung der
Klassen spricht.

\subsection{Auswertung}

\printbibliography

\end{document}
