% Options for packages loaded elsewhere
\PassOptionsToPackage{unicode}{hyperref}
\PassOptionsToPackage{hyphens}{url}
%
\documentclass[
]{article}
\usepackage{amsmath,amssymb}
\usepackage{iftex}
\ifPDFTeX
  \usepackage[T1]{fontenc}
  \usepackage[utf8]{inputenc}
  \usepackage{textcomp} % provide euro and other symbols
\else % if luatex or xetex
  \usepackage{unicode-math} % this also loads fontspec
  \defaultfontfeatures{Scale=MatchLowercase}
  \defaultfontfeatures[\rmfamily]{Ligatures=TeX,Scale=1}
\fi
\usepackage{lmodern}
\ifPDFTeX\else
  % xetex/luatex font selection
\fi
% Use upquote if available, for straight quotes in verbatim environments
\IfFileExists{upquote.sty}{\usepackage{upquote}}{}
\IfFileExists{microtype.sty}{% use microtype if available
  \usepackage[]{microtype}
  \UseMicrotypeSet[protrusion]{basicmath} % disable protrusion for tt fonts
}{}
\makeatletter
\@ifundefined{KOMAClassName}{% if non-KOMA class
  \IfFileExists{parskip.sty}{%
    \usepackage{parskip}
  }{% else
    \setlength{\parindent}{0pt}
    \setlength{\parskip}{6pt plus 2pt minus 1pt}}
}{% if KOMA class
  \KOMAoptions{parskip=half}}
\makeatother
\usepackage{xcolor}
\usepackage[margin=1in]{geometry}
\usepackage{graphicx}
\makeatletter
\def\maxwidth{\ifdim\Gin@nat@width>\linewidth\linewidth\else\Gin@nat@width\fi}
\def\maxheight{\ifdim\Gin@nat@height>\textheight\textheight\else\Gin@nat@height\fi}
\makeatother
% Scale images if necessary, so that they will not overflow the page
% margins by default, and it is still possible to overwrite the defaults
% using explicit options in \includegraphics[width, height, ...]{}
\setkeys{Gin}{width=\maxwidth,height=\maxheight,keepaspectratio}
% Set default figure placement to htbp
\makeatletter
\def\fps@figure{htbp}
\makeatother
\setlength{\emergencystretch}{3em} % prevent overfull lines
\providecommand{\tightlist}{%
  \setlength{\itemsep}{0pt}\setlength{\parskip}{0pt}}
\setcounter{secnumdepth}{5}
\usepackage[german]{babel}
\usepackage{mathtools}
\usepackage{tikz}
\usepackage{pgf}
\usepackage{pgfplots}
\usepackage{csquotes}
\AtBeginDocument{
\renewcommand{\maketitle}{}
}
\PassOptionsToPackage{a4paper,margin = 2.5cm}{geometry}
\usepackage{geometry}
\newcommand{\bcenter}{\begin{center}}
\newcommand{\ecenter}{\end{center}}
\renewcommand{\contentsname}{Inhalt}
\usepackage{blindtext}
\usepackage[backend=biber, style = apa]{biblatex}
\addbibresource{Literatur.bib}
\ifLuaTeX
  \usepackage{selnolig}  % disable illegal ligatures
\fi
\usepackage[]{biblatex}
\addbibresource{Literatur.bib}
\usepackage{bookmark}
\IfFileExists{xurl.sty}{\usepackage{xurl}}{} % add URL line breaks if available
\urlstyle{same}
\hypersetup{
  pdftitle={Test},
  pdfauthor={Franz Andersch \& Niklas Münz},
  hidelinks,
  pdfcreator={LaTeX via pandoc}}

\title{Test}
\author{Franz Andersch \& Niklas Münz}
\date{2024-07-31}

\begin{document}
\maketitle

\tableofcontents
\thispagestyle{empty}
\clearpage
\pagenumbering{arabic}
\section{Einleitung}

Support Vector Machines sind eine feine Sache
\parencite{james_introduction_2021} \blindtext \blindtext

\section{Funktionsweise}

\blindtext\parencite{cortes_support-vector_1995}

\begin{figure}
\begin{tikzpicture}
        \begin{axis}[
            xmin=0,
            xmax=9,
            ymin=0,
            ymax=9,
            axis equal,
            width=15cm,
            axis x line=center,
            axis y line=center,
            xmajorticks=false,
            ymajorticks=false,
            ylabel style={right,font=\huge},
            ylabel=$X_2$,
            xlabel style={above,right,font=\huge},
            xlabel=$X_1$,
            line width=1pt
            ]
            %\addplot[no markers, domain=-1:11]{(5/3)*x};
            \draw[magenta,line width=2.5pt] (axis cs:0,0.05)--(axis cs:3.078235,5.177058);
            \draw[->,green,line width=2.5pt] (axis cs:0,0)--(axis cs:5,8.333) node[above,pos=1,font=\huge]{\textcolor{green}{$\mathbf{\beta}$}};
                \draw[magenta,thick] (axis cs:0,0.05)--(axis cs:3.078235,5.177058);
            \draw[->,line width=2.5pt] (axis cs:0,0)--(axis cs:8,2.2);
            \draw[->,line width=2.5pt] (axis cs:0,0)--(axis cs:4.5,4.3);
            \draw[->,line width=2.5pt] (axis cs:0,0)--(axis cs:1,6.4);
            
            \addplot[no markers,domain=-1:11,line width=2.8pt,blue]{-0.6*x+7};
            \addplot[no markers, domain=-1:11,line width=1.5pt,dashed]{-0.6*x+8};
            \addplot[no markers, domain=-1:11,line width =1.5pt,dashed]{-0.6*x+6};
        \end{axis}
    \end{tikzpicture}
    \caption{Definition der Entscheidungsgrenze}
\end{figure}

\newpage
\addcontentsline{toc}{section}{Literatur}
\newpage
\listoffigures

\printbibliography

\end{document}
