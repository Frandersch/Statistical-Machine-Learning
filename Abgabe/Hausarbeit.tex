% Options for packages loaded elsewhere
\PassOptionsToPackage{unicode}{hyperref}
\PassOptionsToPackage{hyphens}{url}
%
\documentclass[
]{article}
\usepackage{amsmath,amssymb}
\usepackage{iftex}
\ifPDFTeX
  \usepackage[T1]{fontenc}
  \usepackage[utf8]{inputenc}
  \usepackage{textcomp} % provide euro and other symbols
\else % if luatex or xetex
  \usepackage{unicode-math} % this also loads fontspec
  \defaultfontfeatures{Scale=MatchLowercase}
  \defaultfontfeatures[\rmfamily]{Ligatures=TeX,Scale=1}
\fi
\usepackage{lmodern}
\ifPDFTeX\else
  % xetex/luatex font selection
\fi
% Use upquote if available, for straight quotes in verbatim environments
\IfFileExists{upquote.sty}{\usepackage{upquote}}{}
\IfFileExists{microtype.sty}{% use microtype if available
  \usepackage[]{microtype}
  \UseMicrotypeSet[protrusion]{basicmath} % disable protrusion for tt fonts
}{}
\makeatletter
\@ifundefined{KOMAClassName}{% if non-KOMA class
  \IfFileExists{parskip.sty}{%
    \usepackage{parskip}
  }{% else
    \setlength{\parindent}{0pt}
    \setlength{\parskip}{6pt plus 2pt minus 1pt}}
}{% if KOMA class
  \KOMAoptions{parskip=half}}
\makeatother
\usepackage{xcolor}
\usepackage[margin=1in]{geometry}
\usepackage{graphicx}
\makeatletter
\def\maxwidth{\ifdim\Gin@nat@width>\linewidth\linewidth\else\Gin@nat@width\fi}
\def\maxheight{\ifdim\Gin@nat@height>\textheight\textheight\else\Gin@nat@height\fi}
\makeatother
% Scale images if necessary, so that they will not overflow the page
% margins by default, and it is still possible to overwrite the defaults
% using explicit options in \includegraphics[width, height, ...]{}
\setkeys{Gin}{width=\maxwidth,height=\maxheight,keepaspectratio}
% Set default figure placement to htbp
\makeatletter
\def\fps@figure{htbp}
\makeatother
\setlength{\emergencystretch}{3em} % prevent overfull lines
\providecommand{\tightlist}{%
  \setlength{\itemsep}{0pt}\setlength{\parskip}{0pt}}
\setcounter{secnumdepth}{5}
\usepackage[german]{babel}
\usepackage{mathtools}
\usepackage{tikz}
\usepackage{pgf}
\usepackage{csquotes}
\AtBeginDocument{
\renewcommand{\maketitle}{}
}
\PassOptionsToPackage{a4paper,margin = 2.5cm}{geometry}
\usepackage{geometry}
\usepackage{float}
\usepackage{enumitem}
\newcommand{\bcenter}{\begin{center}}
\newcommand{\ecenter}{\end{center}}
\renewcommand{\contentsname}{Inhalt}
\usepackage{blindtext}
\usepackage[backend=biber, style = apa]{biblatex}
\addbibresource{Literatur.bib}
\ifLuaTeX
  \usepackage{selnolig}  % disable illegal ligatures
\fi
\usepackage[]{biblatex}
\addbibresource{Literatur.bib}
\usepackage{bookmark}
\IfFileExists{xurl.sty}{\usepackage{xurl}}{} % add URL line breaks if available
\urlstyle{same}
\hypersetup{
  pdftitle={Hausarbeit},
  pdfauthor={Franz Andersch \& Niklas Münz},
  hidelinks,
  pdfcreator={LaTeX via pandoc}}

\title{Hausarbeit}
\author{Franz Andersch \& Niklas Münz}
\date{2024-08-22}

\begin{document}
\maketitle

\begin{titlepage}
	\begin{center}
		
		\includegraphics[width=0.5\textwidth]{images/UB-Logo-blau.png}
		
		\LARGE
		\textbf{Otto-Friedrich Universität Bamberg}\\
		\textbf{Lehrstuhl für Statistik und Ökonometrie}\\
		\textbf{Statistical Machine Learning}
		
		
		\vspace*{1.5cm}
		
		\Huge
		\textbf{Unser Thema}
		
		
		\vspace{2cm}
		
		\Large
		\begin{minipage}[t]{0.4\textwidth}
			\textbf{Autoren:}
		\end{minipage}
		\hfill
		\begin{minipage}[t]{0.3\textwidth}
			Niklas Münz\newline
			Matrikel Nr.: 2150715
			
			\vspace{0.3cm}
			
			Franz Andersch\newline
			Matrikel Nr.: 2154877
			
		\end{minipage}
		
		\vspace{0.5cm}
		
		\Large
		\begin{minipage}[t]{0.4\textwidth}
			\textbf{Studiengang:}
		\end{minipage}
		\hfill
		\begin{minipage}[t]{0.3\textwidth}
			MSc. Survey Statistics \newline
			\& Data Analysis
		\end{minipage}
		
		\vfill
		Sommersemester 2024
		
	\end{center}
\end{titlepage}
\newpage
\tableofcontents
\thispagestyle{empty}
\clearpage
\pagenumbering{arabic}
\section{Einleitung}
\section{Funktionsweise}

\subsection{Hard Margin Classifier}

Um das grundlegende Prinzip der SVMs darzustellen gehen wir zuerst von
einer Datensituation aus, in der sich zwei gruppen optimal durch eine
lineare entscheidungsgrenze trennen lassen. Das endgültige Ziel ist es
eine sogenannte Hyperplane zu finden die diese Daten möglichst gut
seperiert und als Entscheidungsgrenze funktioniert. Die allgemeine Form
einer solchen hyperplain lautet \begin{align}
\beta_0+ \beta_1 X_1+\beta_2 X_2+...+\beta_n X_n=0\label{eq:hyperebene}
\end{align} oder in Vektorschreibweise \begin{align}
\overline{\beta}\cdot\overline{x}+\beta_0=0 \label{eq:hyperplanevec}
\end{align} Die geometrische Interpretation des Vektors \(\beta\) und
des Skalars \(\beta_0\) wird in Abbildung \ref{fig:Ebene} im zwei
dimensionalen Fall dargestellt.

\begin{figure}[h]
    \centering
    \begin{minipage}{0.45\textwidth} 
        \centering
        \includegraphics[width=\textwidth,trim=0.5cm 0.5cm 0.5cm 0.5cm]{Images/decision_boundary.pdf} 
        \caption{Konstruktion der Hyperebene}
        \label{fig:Ebene}
    \end{minipage}\hfill
    \begin{minipage}{0.45\textwidth} 
        \centering
        \includegraphics[width=\textwidth,trim=0.5cm 0.5cm 0.5cm 0.5cm]{Images/guttes.pdf}
        \caption{Konstruktion der Margins}
        \label{fig:Margin}
    \end{minipage}
\end{figure}

Die blaue Linie soll die Hyperplain darstellen. Im zweidiemnsionalen
handelt es sich hier um eine Linie. Der \(2 \times 1\) Vektor
\(\overline{\beta}\) liegt immer senkrecht zur konstruierten Hyperplane.
Würde man alle Vektoren, die auf der Hyperplane landen, auf den
\(\overline{\beta}\)-Vektor Projezieren, dann hätten alle diese
Projektionen die selbe Länge \(c\). Also gilt für alle Punkte, die auf
der Ebene liegen. \begin{align}
\frac{\overline \beta}{||\overline{\beta}||}\cdot \overline{x}=c \Leftrightarrow \overline{\beta}\cdot \overline{x}=c \cdot ||\overline{\beta}||\label{eq:betanull}
\end{align} ersetzet man in \eqref{eq:betanull}
\(c \cdot ||\overline{\beta}||\) mit \(-\beta_0\) und zieht dies dann
auf die andere Seite, kommen man wieder bei der ursprünglichen Form aus
Formel \eqref{eq:hyperplanevec} raus.

Als nächstes stellt sich jetzt die Frage, welches \(\overline{\beta}\)
und \(\beta_0\) die optimale Hyperplane darstellen. Betrachtet man die
Abbildung \ref{fig:Margin}, dann ist zu erkennen, dass die Datenpunkte
durch die blaue Linie getrennt werden. Allerdings könnte man
theorethisch undendlich viele andere Hyperplains durch rotation oder
verschiebung konstruieren, die trotzdem die Daten in ihren Ausprägungen
trennen. Um eine eindeutige Lösung zu finden, wird als nächstes ein
bereich um die Hperplane abgesteck. In Abbildung \ref{fig:Margin}
dargestell durch die gestrichelten scharzen linien, welche man als
Schranken bezeichnen könnte. In diesem Bereich sollen keine Datenpunkte
liegen und die Schranken sollen immer parallel zur Hyperplain seien und
den gleichen Abstand zu ihr haben. Außerdem dürfen keine positiven
Samples unterhalb der oberen Schranke liegen und keine negativen
unterhalb. Das gegenteil gilt dementsprechend für die untere Schranke.
Als Definition für die beiden Schranken wird festgelegt \begin{align}
\overline{\beta}\cdot \overline{x}+\beta_0=1\label{eq:posSV}
\end{align} für die Schranke in richtung der grünen Datenpunkte und
\begin{align}
\overline{\beta}\cdot \overline{x}+\beta_0=-1\label{eq:negSV}
\end{align} für die Schranke in Richting der roten Datenpunkte. Aus
dieser Beschränkung für die Hyperplane können wir auch ableiten, dass
für die positiven Samples \(\overline{x}^+\) immer gilt
\(\overline{\beta}\cdot \overline{x}^++\beta_0\ge 1\) und für negative
Samples \(\overline{x}^-\) immer gilt
\(\overline{\beta}\cdot \overline{x}^-+\beta_0\le -1\). Durch einführen
einer weiteren Variable \(y\), welche die eigenschaft hat, dass die den
Wert 1 bei einem positiven und den Wert -1 bei einem negativen Sample
annimmt, können diese zwei Beschränkungen zu einer zusammengefasst
werden \begin{align}
y_i(\overline{\beta}\cdot \overline{x}_i+\beta_0)\ge 1\label{eq:Nebenbedingung}
\end{align} Da das Verfahren auch maximum Margin Classifier gennant
wird, gilt es jetzt noch eine Definition für den Margin also den Abstand
zwischen den zwei Schranken zu finden, der schließlich maximiert werden
soll. Damit diese Schranken, maximal weit auseinander liegen, muss es
zwangsläufig Datenpunkte geben, die genau auf den Schranken liegen.
Diese Datenpunkte haben eine wichtige Rolle für die Konstruktion des
Margins. Es sind ausschließlich diese Datenpunkte, die einen Einfluss
auf die finalen werte von \(\overline{\beta}\) und \(\beta_0\) haben
werden. Sie werden \textbf{Support-Vektoren} genannt und geben den SVMs
ihren Namen.

\begin{figure}[h]
\centering
 \includegraphics[width=0.5\textwidth,trim=0.5cm 0.5cm 0.5cm 0.5cm]{Images/margin.pdf} 
        \caption{Abhängigkeit des Margins von den Support-Vektoren}
        \label{fig:SupVecs}
\end{figure}

In Abbildung \ref{fig:SupVecs} sind zwei solcher Support-Vektoren zu
einem negativen und positiven Sample dargestellt. Der Margin kann dann
dargestellt werden als eine Projektion dieser Differenz
(\(\overline{x}^+-\overline{x}^+\)) auf den \(\overline{\beta}\)-Vektor.
Damit am Ende die Länge dieses Margins \(M\) rauskommt muss
\(\overline{\beta}\) noch durch seine Länge geteilt werden.
\begin{align}
M=\frac{\overline{\beta}}{||\overline{\beta}||}\cdot \left(\overline{x}^+-\overline{x}^+\right)=\frac{\overline{\beta}\cdot\overline{x}^- -\overline{\beta}\cdot\overline{x}^+}{||\overline{\beta}||}\label{eq:maring1}
\end{align} Es ist bekannt, dass für postive Supportvektoren gilt
\(\overline\beta \overline{x}^+ +\beta_0 = 1 \Leftrightarrow \overline\beta \overline{x}^+=1-\beta_0\)
und für negative
\(\overline\beta \overline{x}^- +\beta_0 = -1 \Leftrightarrow \overline\beta \overline{x}^-=-1-\beta_0\).
Setzt man dies ein in \eqref{eq:maring1}, erhält man als
Maximisierungsziel \begin{align}
M=\frac{1-\beta_0-(-1-\beta_0)}{||\overline\beta||}=\frac{2}{||\overline\beta||}\label{eq:margin2}
\end{align} Um diesen Maximierungsschritt angenehmer zu gestalten, wird
an der Stelle versucht den Ausdruck
\(\frac{1}{2}||\overline \beta||^2=\frac{1}{2}\overline \beta '\cdot \overline \beta\)
zu minimieren. Was im Endeffekt ebenfalls dazu führt, dass der Ausdruck
in \eqref{eq:margin2} maximiert wird.

Dieses Optimierungsproblem mit der Nebenbedingung aus Formel
\eqref{eq:Nebenbedingung} lässt sich am besten über Lagrange-Multiplier
lösen \begin{align}
\mathcal{L}(\overline\beta,\beta_0,\overline \alpha)=\frac{1}{2}\overline \beta ' \overline \beta-\sum \alpha_i[y_i(\overline \beta \cdot \overline{x_i}+\beta_0)-1]\label{eq:Lagrange}
\end{align} Wird dieser Ausdruck partiell abgeleitet und gleich null
gesetzt erhält man als zwischen ergebnis \begin{align}
\frac{\partial \mathcal{L}}{\partial \beta}=\beta-\sum \alpha_i y_i \overline{x_i}\overset{!}{=}0 \Rightarrow \beta=\sum \alpha_i y_i \overline{x_i}\label{eq:solutbeta}
\end{align} somit zeigt sich, dass \(\overline{\beta}\) als
linearkombination der Inputvektoren dargestellt werden kann. Weiterhin
gilt für \(\beta_0\) \begin{align}
\frac{\partial \mathcal{L}}{\partial \beta_0}=\sum \alpha_i y_i \overline{x_i}\overset{!}{=}0\label{eq:solutbeta0}
\end{align} Setzt man dies in \eqref{eq:Lagrange} ein erhält man einen
neuen Ausdruck, denn es gilt zu minimieren \begin{align}
\mathcal{L}(\overline \alpha)=-\frac{1}{2}\sum \sum \alpha_i \alpha_j y_i y_j \overline{x}_i \cdot \overline{x}_j+\sum \alpha_i\label{eq:dualproblem}
\end{align} Die Lösung für diesen Ausdruck erfolgt dann über sogenannte
\enquote{standard non linear optimization algorithms for quadratic forms}
\parencite{boserTrainingAlgorithmOptimal1992}. Nachem für
\(\overline \alpha\) gelöst wurde, kann dies in \eqref{eq:solutbeta}
eingesetzt werden um das optimale \(\overline{\beta}\) zu erhalten. Es
kann gezeigt werden, dass die gelösten \(\alpha_i\) lediglich für die
Supportvektoren werte ungelich Null annehmen. Somit ist der
Koeffizientenvektor \(\overline{\beta}\) sogar eine Linearkombination
von nur den Supportvektoren
\parencite{boserTrainingAlgorithmOptimal1992}. Die letzte unbekannte
\(\beta_0\) kann gelöst werden, indem man mithilfe von einem
positiven/negativen Support Vektor \eqref{eq:posSV}/ \eqref{eq:negSV}
nach \(\beta_0\) löst.

Mit den gelösten Werten zur optimalen Hyperplane kann jetzt auch eine
Entscheidungsregel für ungelabelte Datenvektoren \(\overline{x}_u\)
konstruiert werden. Bedenkt man also wenn man einen Vektor der nicht auf
der Hypeplane liegt in \eqref{eq:betanull} einsetzt erhält man also
\(\frac{\overline \beta}{||\overline{\beta}||}\cdot \overline{x}_u=c+k\).
Wenn \(k\) positiv ist, liegt der neue Datenpunkt oberhalb der
Hyperplane liegt und somit als positives Sample gewertet wird. Wenn
\(k\) negativ ist, dann liegt der Datenpunkt unterhalb der Hyperplane
und wird als negativ gewertet. Mit der gleichen Umformung wie weiter
oben schon beschrieben kommt man zu folgender Entscheidungsregel
\begin{align}
f(\overline{x}_u)=\begin{cases}\mathrm{positiv}&\text{wenn } \overline{\beta}\cdot \overline{x}_u+\beta_0 > 0\\
\mathrm{negativ} & \text{wenn }\overline{\beta}\cdot \overline{x}_u+\beta_0<0
\end{cases}\label{eq:decisionf}
\end{align}

\subsection{Soft Margin Classifier}

Dass die Daten sich perfekt linear trennen lassen ist zwar ein gut um
die Vorgehensweise zu veranschaulichen, tritt aber in realen Situationen
so gut wie nie auf. Falls sich postitive und negative Samples im Raum
überlappen, ist die Konstruktion einer Hyperplane wie beim Hard Margin
Classifier unmöglich. Man müsste also entweder auf eine nicht lineare
Hypeplane ausweichen oder man erweicht die vorgaben für die Konstruktion
der Hyperplane. Zweiteres ist genau das, was durch die Soft Margin
Classifier erreicht wird. Die Vorgabe für die Konstruktion der Schranken
ermöglicht es einzelnen Datenpunkten auf der falschen Seite der
Schranke, ja sogar der Entscheidungsgrenzen zu liegen. Dafür wird für
die Einschränkungen eine sogennannte Slackvariable \(\varepsilon\)
eingeführt \parencite{jamesIntroductionStatisticalLearning2021}. Setzt
man diese in diese in \eqref{eq:Nebenbedingung} lautet die neuen
Nebenbedingung \begin{align}
y_i(\beta \cdot \overline{x}_i-\beta_0)>1- \varepsilon_i \label{eq:nebbedsfm}
\end{align}

\begin{figure}[h]
\centering
\includegraphics[width=0.5\textwidth,trim=0.5cm 0.5cm 0.5cm 0.5cm]{Images/slackvariable.pdf} 
        \caption{Funktion der Slack Variable}
        \label{fig:slackvariable}
\end{figure}

Jetzt könnte man versuchen diese neue Nebenbedingung einfach in das
zuvor angewandte Optimisierungsverfahren einzufügen. Allerdings besteht
hier das Problem, dass \(\varepsilon\) einfach immer maximal groß
gewählt wird und so die Bedingung immer erfüllt wird. Um das Ausmaß der
Verletzung der usptünglichen Annahmen zu begrenzen, aber trotzdem noch
gewisse Abweichung zuzulassen, wird ein weitere Parameter \(C\)
eingeführt, als regularisierender Parameter für \(\varepsilon\). Leitet
daraus zusammen mit der Restriktion \(\varepsilon\ge0\) wieder einen
Lagrangefunktion her erhält man \begin{align}
\mathcal{L}(\overline\beta,\beta_0,\overline\alpha,\overline\varepsilon,\overline\lambda)=\frac{1}{2}\overline\beta'\cdot \overline\beta + C \sum_{i=1}^{n}\varepsilon_i-\underbrace{\sum \alpha_i[y_i(\overline\beta \cdot \overline{x_i}+\beta_0)-1+\varepsilon_i]}_{\text{für }y_i(\overline\beta \cdot \overline{x}_i-\beta_0)>1- \varepsilon_i}-\underbrace{\sum \lambda_i \varepsilon_i }_{\substack{\text{für}\\ \varepsilon_i \ge 0}}
\end{align} Wenn dieser Ausdruck wie beim Hardmargin Classifier gelöst
wird und die Ergebnisse eingesetzt werden erhält man wieder den Ausruck
aus \eqref{eq:dualproblem} mit der zusätzlichen Einschränkung
\(0\le \alpha_i \le C\). Dieses Maximierungsproblem wird dann genauso
aufgelöst wie bei dem Hard Margin Classifier und die Entscheidungsregel
ist ebenfalls gleich.

\subsection{Der Kernel Trick}

Auch wenn eine lineare Entscheidungsgrenze Vorteile in Sachen
Generalisierbarkeit bietet, ist sie doch nicht für jede Datensituation
geeignet.In Abbildung \ref{fig:nonlinearsep} ist es sehr gut zu
erkennen, dass in diesem Fall eine lineare Grenze zwischen den Klassen
keinen Sinn Ergeben würde und eine elliptische Form wahrscheinlich
besser geeignet wäre. Eine Lösung für dieses Problem, wäre den
Merkmalsraum zu erweitern. So könnte die angenommene Formel für die
lineare Hyperebene in \eqref{eq:hyperebene} durch polynomterme der
Merkmale \(X_i\) oder durch Interaktionstherme erweitert werden. Dies
führt dazu, dass die Entscheidungsgreze in diesem vergrößerten
Merkmalsraum immer noch linear ist, aber die Trennung möglich ist (siehe
Abbildung \ref{fig:featurexten}). Transformiert man diese dann wieder in
den ursprünglichen Merkmalsraum ist die Entscheidungsgrenze dann nicht
mehr linear. Allerdings führt diese Herangehensweise zu eine starken
Anstieg des Rechenaufwands, da die Möglichkeiten der Merkmalserweiterung
endlos sind \parencite{jamesIntroductionStatisticalLearning2021}.

Die Lösung für das Problem sind sogenannte Kernel Funktionen. Betrachtet
man die entscheidungsfunktion \eqref{eq:decisionf} und setzt für
\(\overline{\beta}\) die Gleichung aus \eqref{eq:solutbeta} erhält man
\begin{align}
f(x_u)= \sum \alpha_i y_i x_i \cdot x_u +\beta_0
\end{align} Es zeigt sich also, dass die Entscheidungsfunktion im
Wesentlichen aus einer Linearkombiantion von Punktprodukten aus dem
Vektor \(x_u\) mit allen Trainingsvektoren \(x_i\) ergibt. Diese
Punktprodukt kann als Ähnlichkeitsmaß zwischen dem neuen Datenpunkt und
dem jeweiligen Trainingsdatenpunkt interpretiert werden. Es ist nun
Möglich diese Produkte durch eine Funktion zu ersetzen, welche die
Ähnlichkeiten von Datenpunkten anders bewertet. Diese sogenannte Kernel
Funktion \(K(x_i,x_j)\) ermöglicht es eine flexiblere
Entscheidungsgrenze zu implementieren. Der Vorteil ist dabei, dass die
Kernel Funktion nur auf alle Punktprodukte angewndet wird und es dabei
nicht nötig ist den Merkmalsraum zu Erweitern, was wiederum Rechenzeit
spart\parencite{jamesIntroductionStatisticalLearning2021}. Die
Entscheidungsfunktion wird dann mithilfe dieser Kernelfunktionen
berechnet: \begin{align}
  f(x_u)=\sum \alpha_i y_i K(x_i,x_u)+\beta_0
\end{align}

\begin{figure}[h]
    \centering
    \begin{minipage}{0.45\textwidth} 
        \centering
        \includegraphics[width=\textwidth,trim=0.5cm 0.5cm 0.5cm 0.5cm]{Images/nonlinearseperable.pdf} 
        \caption{nicht linear getrennte Daten}
        \label{fig:nonlinearsep}
    \end{minipage}\hfill
    \begin{minipage}{0.45\textwidth} 
        \centering
        \includegraphics[width=\textwidth,trim=0.5cm 0.5cm 0.5cm 0.5cm]{Images/featurexpansion.pdf}
        \caption{Feature Erweiterung}
        \label{fig:featurexten}
    \end{minipage}
\end{figure}

Es gibt eine ganze Reihe an Kernelfunktionen, die bei SVMs Anwedung
finden. Die Grundlage ist der lineare Kernel, wobei dieser lediglich das
Punktprodukt beschreibt, also praktisch genau das macht, was bei einer
linearen Entscheidungsgrenze gemacht wird. Deweiteren gibt es den
Polynomial Kernel. Dieser hat die Form \begin{align}
 K(x_i,x_u)=(1+x_i\cdot x_u)^d
\end{align} Die Verwendung von diesem Kernel führt dazu, dass die
Entscheidungsgrenze sich ähnlich Verhält, wie als würde man zu Beginn
eine Merkmalserweiterung mit Polynomen vom Grad \(d\) durchführen (siehe
Abbildung \ref{fig:polykernel}. Eine weitere Kernel Funktion ist der
Radial Basis Function Kernel(RBF) mit der Form \begin{align}
K(x_i,x_u)=\exp\left(-\gamma ||x_i-x_u||^2\right)
\end{align} Für diesen Kernel wird die quadrierte euklidische Distanz
als Ähnlichkeitsmaß verwendet, was dazu führt, dass für diejenigen
\(x_i\) die näher an \(x_u\) liegen, in der Entscheidungsfunktion einen
größeren Einfluss haben. Der Parameter \(\gamma\) legt dann fest, wie
stark der Einfluss der Distanz sein soll. Die projektion die der Kernel
hier macht ist eine, in einen uendlich großen Merkmalsraum . Daher
könnte man selbst durch vorgeriges Erweitern des Merkmalsraums nicht das
Ergebnis eines RBF Kernel replizieren und dies führt auch zu einer sehr
flexiblen Entscheidungsgrenze (siehe Abbildung \ref{fig:radialkernel})\\

\begin{figure}[H]
    \centering
    \begin{minipage}{0.45\textwidth} 
        \centering
        \includegraphics[width=\textwidth,trim=0.5cm 0.5cm 0.5cm 0.5cm]{Images/ploynomial kernel.pdf} 
        \caption{Mögliche Entscheidungsgrenze für polynomial Kernel}
        \label{fig:polykernel}
    \end{minipage}\hfill
    \begin{minipage}{0.45\textwidth} 
        \centering
        \includegraphics[width=\textwidth,trim=0.5cm 0.5cm 0.5cm 0.5cm]{Images/radial_kernel.pdf}
        \caption{Mögliche Entscheidungsgrenze für RBF Kernel}
        \label{fig:radialkernel}
    \end{minipage}
\end{figure}

Es gibt noch einen Reihe wietere Kernel, die auf unterschiedlichen
Ähnlichkeitsmaßen beruhen, aber eher seltener oder nur in speziellen
Zusammenhänge angewendet werden. Wichtig ist anzumerken, dass die
Verwendung von Kernels zwar die Flexibilität der Entscheidungsgrenze
erhöht, damit aber auch die Gefahr von overfitting einhergeht.
Zusätzlich werden mit den Kernels auch neue Hyperparameter wie \(d\)
oder \(\gamma\) eingeführt, die bei der Model selektion ebenfalls
beachtet werden müssen.\\

\section{Vor- und Nachteile der Methode}

Support Vector Machines haben ein hohes Ansehen unter den Machine
Learning Algorithmen, da sie einige Vorteile mit sich bringen. Aufgrund
der Idee einer Soft Margin und des ``Kernel Tricks'' ist die Methode
sehr flexibel und kann für spezielle Anwendungsbereiche angepasst werden
\parencite{bennettSupportVectorMachines2000}.Dazu sind die Ergebnisse
stabil und reproduzierbar, was sie von anderen Methoden wie
beispielsweise Neural Networks abhebt. Auch die Anwendung ist
vergleichsweise einfach, da es eine überschaubare Anzahl an Parametern
gibt (wie beispielsweise bei der SVM mit radialem Kern nur der gamma-
und cost-Parameter festzulegen ist).\newline Durch die Möglichkeit der
Nutzung verschiedener Kerne sind SVM´s überaus vielseitig. Die Auswahl
des Kerns ermöglicht es äußerst flexible Entscheidungsgrenzen zu formen
\parencite{kuhnAppliedPredictiveModeling2013}. Dadurch können SVM´s an
verschiedene Datensituationen angepasst werden.\newline Ein weiterer
Vorteil ist, dass die Methode weitgehend robust gegenüber overfitting
ist \parencite{kuhnAppliedPredictiveModeling2013}. Dafür verantwortlich
ist der Cost-Parameter, anhand dessen der Fit an die Daten kontrolliert
werden kann. Jedoch birgt dies auch Probleme (Erläuterungen im folgenden
Abschnitt).\newline Diese Vorteile resultieren in einer allgemein
häufigen Nutzung von SVM´s in der Wissenschaft. Sie haben folglich
bewiesen, dass sie für verschiedenste Aufgaben gut funktionieren
\parencite{kuhnAppliedPredictiveModeling2013}.

Trotz der vielfachen Nutzung von SVM´s, bringen sie auch Nachteile mit
sich. Das wohl größte Problem liegt in der Modellselektion
\parencite{bennettSupportVectorMachines2000}. Wie bereits im vorherigen
Abschnitt erwähnt, ist die Auswahl der Parameter von hoher Bedeutung bei
der Performance und dem Fit an die Daten. So kontrollieren die
Kernspezifischen Parameter und der Cost-Parameter einerseits die
Komplexität und andererseits den Fit an die Daten
\parencite{kuhnAppliedPredictiveModeling2013}. Dabei kann die Wahl der
Parameter sowohl zu einem underfit als auch zu einem overfit führen.
Jedoch haben nicht nur die Parameter einen Einfluss auf die Performance
sondern bereits die Wahl des Kerns kann entscheidend sein
\parencite{burgesTutorialSupportVector1998}. Je nach Datensituation
können SVM´s mit verschiedenen Kernen äußert unterschiedliche Ergebnisse
liefern. Dies zeigt die Sensitivität der Methode gegenüber der Wahl des
Kerns und der Parameterabstimmung.\newline Ein weiterer Nachteil ist,
dass die Methode weniger intuitiv und aufwendiger anzuwenden ist als
andere Algorithmen \parencite{bennettSupportVectorMachines2000}. So ist
es zum Beispiel schwer Informationen aus Support Vektoren zu ziehen und
es gibt keine Koeffizienten die interpretiert werden können.\newline
Zuletzt ist zu erwähnen, dass die Methode bei einer hohen Anzahl an
Beobachtungen besonders rechenintensiv ist. So konnte beispielsweise
gezeigt werden, dass insbesondere die SVM mit polynomialem und radialem
Kern eine hohe Rechenzeit aufweisen
\parencite{scholzComparisonClassificationMethods2021}. Dabei konnten
andere Methoden wie die logistische Regression oder k-nearest Neighbour
deutlich besser abschneiden. Dies liegt daran, dass die Lösung des
SVM-Optimierungsproblems die Behebung eines quadratischen
Programmierungsproblems erfordert. Da die Anzahl der zu optimierenden
Parameter mit der Anzahl der Daten quadratisch zunimmt, führt dies zu
einer hohen Rechenkomplexität
\parencite{kecmanSupportVectorMachines2005}.

\section{Daten}

Das Ziel dieser Arbeit ist es, in verschiedenen Datensituationen die
performance von SVM Algorithmen für die binäre Klassifikation zu
evaluieren. Dafür wollen wir eine Reihe von Datensätzen mit
verschiedenen Charakteristiken synthetisch erstellen und entsprechen als
Trainigsdaten verwenden. Die Datensätze unterscheiden sich in zwei
zentralen Eigenschaften. Das erste sind die Dimensionen. Es soll
unterschieden werden in drei Kategorien. Die erste ist, dass es deutlich
mehr Beobachtungen als Variablen gibt, also \(n \gg p\). Ein solches
Datenszenario kann z.B. im Kontext des Zensus auftreten, in dem eine
große Anzahl an Personen befragt wird, aber die Vorgabe besteht, dass
die Bürger nicht zu stark belastet werden sollen, weshalb nur einige
wenige Kernfragen gestellt werden. Das zweite Szenario stellt Datensätze
vor die etwa gleich viele Beobachtung wie Variablen haben
\(n \approx p\). Ein solches Szenario kann in vielen Kontexten
auftreten. Das letzte Szenario behandelt dann entsprechend den Fall
\(n \ll p\). Dies tritt oft im Kontext von Datenerhebungen im
medizinischen Bereich auf, da sehr viele Erhebungen zu kostspielig
wären. Auch im Bereich des Natural Language Processing sind solche
Datensätze häufiger anzutreffen
\parencite{scholzComparisonClassificationMethods2021}.

Die zweite Charakteristik ist der Datengenerierende Prozess. Da in
dieser Arbeit SVMs im Vordergrund stehen und wir hier vor allem Zeigen
wollen, wie SVMs funktionieren, wird der DGP so aufgebaut, dass er der
grundlegenden Idee der SVMs am ehesten Entspricht. Die Vorgehensweise
ist, im ersten Schritt eine Hyperplane, im \(p\)-Dimensionalen Raum, in
einer bestimmte Form zu erstellen und anschließend auf jeweils einer
Seite dieser Hyperplane \(n/2\) zufällige Punkte zu samplen, die die
jeweilige Ausprägung in der Zielvariable repräsentieren.

Insgesammt gibt es auch hier wieder 3 Kategorien. Die erste sind linear
getrennte Daten. Dafür wird eine lineare Hyperplane der Form
\begin{align*}
\beta_0+\beta_1 X_1+\beta_2 X_2 +...+\beta_p X_p=0
\end{align*} mit zufälligen Koeffizienten erzeugt. Diese Daten sollen
also den Annahmen entsprechen, die für SVMs mit linearem Kernel gelten.
Nachem für eine Beobachtung \(j\) zufällig ein Punkt auf der Ebene
gesampelt wurde, wurde dieser andschließend verschoben. Die Verschiebung
erfolgte über eine Skalierung des normierten Normalenvektors
\(k\left(\frac{\overline{\beta}}{||\overline{\beta}||}\right)\). \(k\)
ist dabei eine Zufallszahl mit Mittelwert \(\mu_k\) und Varianz
\(\sigma^2_k\). Dieser Prozess wird \(n/2\) mal wiederholt für die eine
Ausprägung der Zielvariable und dementsprechen \(n/2\) mal für die
andere Ausprägung, dann aber mit \(-\mu_k\) als Mittelwert für \(k\).
\newline In der zweiten Situation hat die Hyperplane eine quadratische
Form: \begin{align*}
\beta_0+\beta_1 X_1 + \beta_2 X_1^2+\beta_3 X_2+\beta_4 X_2^2+...+\beta_{2p-1}X_p+\beta_{2p}X_p^2=0
\end{align*} diese Form der Trennung stellt also eine
Merkmalserweiterung um quadratische Terme dar und funktioniert damit
ähnlich wie eine SVM mit polynomialen Kernel mit \(d=2\). Der letzte DGP
geht von einer noch Komplexeren Entscheidungsgrenzen aus. Es wird hier
ein Hypershpäre im \(p\) dimensionalen Raum erstellt und einmal
innerhalb und einmal außerhalb dieser gesampelt. Dafür wurde für eine
Beobachtung \(j\), \(p-1\) Winkel \(\theta\) zufällig erstellt, ein
Radius \(r\) festgelegt und anschließend die einzelnen Werte
\(X_{1,j},X_{2,j},...,X_{p,j}\) berechnet. Die berechnung erfolgt dabei
über die Definition von sphärischen Koordinaten: \begin{align*}
        X_{1,j} &= r \cos(\theta_1)\\
        X_{2,j} &= r \sin(\theta_1)\cos(\theta_2)\\
        X_{3,j} &= r \sin(\theta_1)\sin(\theta_2)\cos(\theta_3)\\
        &\quad \vdots\\
        X_{p-1,j}&=r \sin(\theta_1)\ldots \sin(\theta_{p-2})\cos(\theta_{p-1})\\
        X_{p,j}&=r \sin(\theta_1)\ldots \sin(\theta_{p-2})\sin(\theta_{p-1})
    \end{align*} Dieser Vorgang wird dann \(n/2\) mal, mit einem
zufälligen Radius mit Mittelwert \(\mu_r\) und einer Varianz
\(\sigma^2_r\) für die eine Ausprägung der Zielvariable wiederholt. Für
die andere Ausprägung wurde das gleiche dann durchgeführt mit einem
neuen Mittelwert \(\mu_r+k,k\in \mathbb{R}\). je nachdem wie \(k\) und
\(\sigma^2_r\) gewählt werden kann die Trennbarkeit der Daten angepasst
werden.

Es ergeben sich daher 9 unterschiedliche Datensituationen, welche in
ihren Dimensionen und Komplexität der Entscheidungsgrenze variieren. Die
Kürzel für die Situationen sind in Tabelle \ref{tab:datensituationen}
abgetragen

\begin{table}[H]
\begin{center}
\begin{tabular}{ |c|c|c|c| }
 \hline
  & linear & polynomial & radial \\
 \hline
 $p \ll n$ & S1 & S2 & S3 \\
 \hline
 $p \approx n$ & S4 & S5 & S6 \\
 \hline
 $p \gg n$ & S7 & S8 & S9 \\
 \hline
\end{tabular}
\end{center}
\caption{Datensituationen}
\label{tab:datensituationen}
\end{table}

Zusätzlich soll nicht nur ein Vergleich zwischen der Performance der
SVMS mit verschiedenen Kernel gemacht werden, sondern auch die
Unterschiede in der Klassifikationsgüte der SVMs zu anderen gängigen
Klassifikationsmethoden gezeigt werden. Dafür werden Regularized
Logistic Regression und k-nearest Neighbour als Vergleichsalgorithmen
hinzugezogen.

\section{Hypothesen}

Im Folgenden werden Studien hinzugezogen, um eine Einschätzung der
Performance in den verschiedenen Szenarien vorzunehmen und Hypothesen
abzuleiten. Vorab ist zu erwähnen, dass die Evaluation von
Klassifikationsmethoden anhand synthetischer Datensätze in der Literatur
begrenzt ist. Da für diese Arbeit die Form der Entscheidungsgrenze
entscheidend ist, werden dennoch auschließlich Arbeiten mit
synthetischen Datensätzen zu Rate gezogen.\newline Aufgrund dessen, dass
der Datengenerierende Prozess hier so ausgearbeitet wurde, dass er mit
den Annahmen der SVMs arbeitet, erwarten wir zuerst einmal eine bessere
Performance der SVM Classifier im Vergleich zu den anderen Methoden.

\begin{minipage}{0.9\linewidth}
\begin{itemize}[leftmargin=0.1\linewidth]
\item[\textbf{H1:}] Die SVM Classifier Performen über alle Datensituationen im Durchschnitt besser als die anderen Classifier
\end{itemize}
\end{minipage}

Des Weiteren wurden in den einzelnen Kategorien des daten generierenden
Prozesses die Entscheidungsgrenzen speziell auf verschiedene Kernels der
SVMs zugeschnitten. Daher sollten SVMs mit linearem Kernel im Setting
mit linearer Entscheidungsgrenze mindestens so gut oder besser als die
restlichen Classifier performen. Gleiches gilt für SVMs mit polynomialen
Kernel im Setting mit einer quadratischen Entscheidungsgrenze und
radiale Kernel bei einer Hypershäre als Entscheidungsgrenze.

\begin{minipage}{0.9\linewidth}
\begin{itemize}[leftmargin=0.1\linewidth]
\item[\textbf{H2:}] Die SVM Classifier mit dem Kernel, der für die jeweiligen DGP zugeschnitten ist sollten mindestens genauso gut oder besser Performen als die restlichen Classifier
\end{itemize}
\end{minipage}

Es konnte weiterhin gezeigt werden, dass in einem Szenario, indem
erheblich mehr Beobachtungen als Dimensionen und eine lineare
Entscheidungsgrenze vorliegen (S1), deutliche Unterschiede zwischen SVM,
k-NN und logistischer Regression bei der Diskriminationsfähigkeit
auftreten
\parencite{entezari-malekiComparisonClassificationMethods2009}. k-NN und
lineare SVM zeigen AUC-Werte nahe 1 auf, was für eine nahezu perfekte
Differenzierung der Klassen spricht. Die logistische Regression hingegen
hat einen Wert knapp über 0.5, was nur etwas besser als eine
Zufallsauswahl ist. Darüber hinaus ist festzustellen, dass die
Unterschiede deutlicher werden, je höher die Anzahl an Beobachtungen
ist.\newline Für den Fall einer radialen Entscheidungsgrenze (S3) sind
die Ergebnisse ähnlich. So erreicht in diesem Beispiel eine SVM mit
radialem Kernel im Vergleich zu einer logistischen Regression eine um
34\% höhere Genauigkeit
\parencite{faveroClassificationPerformanceEvaluation2022}.

Die Szenarien S4 bis S6 finden in der Literatur kaum Beachtung, weshalb
hier keine Studien herangezogen werden können. Liegt jedoch ein Szenario
vor, indem die Anzahl der Dimensionen erheblich größer ist, als die
Anzahl der Beobachtungen, mit einer linearen Entscheidungsgrenze (S7),
sind die Ergebnisse differenzierter zu betrachten. So schneidet die SVM
mit polynomialem Kern am besten unter den genannten Algorithmen ab,
jedoch die lineare SVM am schlechtesten (als Kriterium wurde die
mittlere Performance über 100 Datensätze evaluiert)
\parencite{scholzComparisonClassificationMethods2021}. Während k-NN auch
in diesem Szenario eine gute Performance hat, schneiden logistische
Regression und SVM mit radialem Kern mittelmäßig ab. Hierbei ist wichtig
zu erwähnen, dass in der Studie keine Ergebnisse über die genaue
Performance präsentiert wurden, sondern lediglich die Ränge der 25
behandelten Klassifikationsmethoden. Somit können nur eingeschränkte
Schlussfolgerungen gezogen werden.

Basierend auf den Ergebnissen der genannten Studien können folgende
Schlussfolgerungen gezogen werden.

\begin{minipage}{0.9\linewidth}
\begin{itemize}[leftmargin=0.1\linewidth]
\item[\textbf{H3:}] In niedrigdimensionalen Szenarien performen k-NN und SVM´s besser als eine logistische Regression.
\end{itemize}
\end{minipage}

Jedoch ist zu vermuten, dass die Wahl des Kerns bei SVM´s einen großen
Einfluss auf die Performance hat. So ist, basierend auf den
mathematischen Grundlagen, anzunehmen, dass die SVM mit dem jeweils
passenden Kern zu der vorliegenden Datensituation am besten performt. Da
die SVM mit polynomialem und radialem Kern weitaus flexibler sind,
werden diese voraussichtlich insgesamt betrachtet besser abschneiden als
die SVM mit linearem Kern.\newline

In hochdimensionalen Szenarien zeigt vermutlich die SVM mit polynomialem
oder radialem Kern eine gute Performance, unabhängig von der Form der
Entscheidungsgrenze, während die lineare SVM voraussichtlich weniger gut
abschneiden wird. Es scheint so, dass auch k-NN und logistische
Regression in hochdimensionalen Szenarien zumindest mittelmäßig
abschneiden. Es ist aber auch bekannt, dass gerade die k-NN Methode in
hochdimensionalen Settings schlechter performt
\parencite{jamesIntroductionStatisticalLearning2021}. Hier ist jedoch zu
beachten, dass nur eine lineare Entscheidungsgrenze betrachtet wurde und
in den Szenarien S8 und S9 andere Ergebnisse möglich sind. Wir schließen
Final daraus:

\begin{minipage}{0.9\linewidth}
\begin{itemize}[leftmargin=0.1\linewidth]
\item[\textbf{H4:}] In hochdimensionalen Settings performen v.a. SVMs mit radialen und polynomialen Kernel besser als die anderen Klassifikationsmethoden
\end{itemize}
\end{minipage}
\section{Ergebnisse}

\subsection{Analyse}

Bevor die Ergebnisse dargelegt werden, wird kurz auf die Vorgehensweise
bei der Analyse eingegangen. In die Analyse werden fünf Modelle
einbezogen: SVM mit linearem, polynomialem und radialem Kern, sowie
regularisierte logistische Regression und k-nearest neighbours.\newline
Vor dem erstellen der Modelle wird ein Tuning der Hyperparameter je
Modell durchgeführt. Dafür wird die Bayesian Optimization Methode
genutzt, welche ein iterativer Algorithmus ist. Hierbei werden die
nächsten Evaluierungspunkte basierend auf zuvor beobachteten Ergebnissen
bestimmt \parencite{yangHyperparameterOptimizationMachine2020}. Der
Algorithmus basiert auf zwei Hauptkomponenten: einem Surrogatmodell und
einer Akqusitionsfunktion. Das Surrogatmodell, wofür hier ein Gaussian
Process genutzt wird, passt die bisher beobachteten Punkte an die
Zielfunktion an. Die Akquisitionsfunktion wählt dann die nächsten Punkte
aus, wobei ein Gleichgewicht zwischen der Erkundung neuer Bereiche und
der Nutzung vielversprechender Regionen angestrebt wird. Dafür wird hier
der Ansatz des Upper-Confidence-Bound genutzt, welcher obere
Konfidenzgrenzen nutzt um den Verlust gegenüber der besten möglichen
Entscheidung, während der Optimierung zu minimieren
\parencite{snoekPracticalBayesianOptimization2012}. Die Bayesian
Optimization wird genutzt, da sie eine schnelle Konvergenz für stetige
Hyperparameter aufweist
\parencite{yangHyperparameterOptimizationMachine2020}. Als
Evaluierungskriterium wird die Genauigkeit der Modelle, welche durch den
Anteil der korrekt klassifizierten Beobachtungen wiedergegeben wird,
verwendet.\newline Basierend auf den Ergebnissen des Tuning werden die
oben genannten Modelle erstellt. Daraufhin werden die Genauigkeit und
die Receiver Operating Characteristic Kurve (ROC-Kurve) bzw. der Area
Under The Curve Wert (AUC-Wert) für jedes Modell bestimmt. Die ROC-Kurve
ist eine grafische Darstellung der Leistungsfähigkeit eines
Klassifikationsmodells, wobei die Sensitivität gegen die Spezifität
abgetragen wird \parencite{fawcettIntroductionROCAnalysis2006}. Die
Sensitivität gibt den Anteil der korrekt als positiv (hier
gleichbedeutend mit Klasse 1) klassifizierten Beobachtungen an während
die Spezifität den Anteil der korrekt als negativ (hier gleichbedeutend
mit Klasse 2) klassifizierten Beobachtungen angibt. Der AUC-Wert bezieht
sich auf die Fläche unterhalb die Kurve und liegt somit im Intervall
{[}0,1{]}, wobei ein Wert von 1 für eine perfekte Klassifikation
spricht, während ein Wert von 0.5 für eine rein zufällige Zuordnung der
Klassen spricht.

\subsection{Auswertung}

\newpage
\addcontentsline{toc}{section}{Literatur}

\printbibliography

\end{document}
