% Options for packages loaded elsewhere
\PassOptionsToPackage{unicode}{hyperref}
\PassOptionsToPackage{hyphens}{url}
%
\documentclass[
]{article}
\usepackage{amsmath,amssymb}
\usepackage{iftex}
\ifPDFTeX
  \usepackage[T1]{fontenc}
  \usepackage[utf8]{inputenc}
  \usepackage{textcomp} % provide euro and other symbols
\else % if luatex or xetex
  \usepackage{unicode-math} % this also loads fontspec
  \defaultfontfeatures{Scale=MatchLowercase}
  \defaultfontfeatures[\rmfamily]{Ligatures=TeX,Scale=1}
\fi
\usepackage{lmodern}
\ifPDFTeX\else
  % xetex/luatex font selection
\fi
% Use upquote if available, for straight quotes in verbatim environments
\IfFileExists{upquote.sty}{\usepackage{upquote}}{}
\IfFileExists{microtype.sty}{% use microtype if available
  \usepackage[]{microtype}
  \UseMicrotypeSet[protrusion]{basicmath} % disable protrusion for tt fonts
}{}
\makeatletter
\@ifundefined{KOMAClassName}{% if non-KOMA class
  \IfFileExists{parskip.sty}{%
    \usepackage{parskip}
  }{% else
    \setlength{\parindent}{0pt}
    \setlength{\parskip}{6pt plus 2pt minus 1pt}}
}{% if KOMA class
  \KOMAoptions{parskip=half}}
\makeatother
\usepackage{xcolor}
\usepackage[margin=1in]{geometry}
\usepackage{graphicx}
\makeatletter
\def\maxwidth{\ifdim\Gin@nat@width>\linewidth\linewidth\else\Gin@nat@width\fi}
\def\maxheight{\ifdim\Gin@nat@height>\textheight\textheight\else\Gin@nat@height\fi}
\makeatother
% Scale images if necessary, so that they will not overflow the page
% margins by default, and it is still possible to overwrite the defaults
% using explicit options in \includegraphics[width, height, ...]{}
\setkeys{Gin}{width=\maxwidth,height=\maxheight,keepaspectratio}
% Set default figure placement to htbp
\makeatletter
\def\fps@figure{htbp}
\makeatother
\setlength{\emergencystretch}{3em} % prevent overfull lines
\providecommand{\tightlist}{%
  \setlength{\itemsep}{0pt}\setlength{\parskip}{0pt}}
\setcounter{secnumdepth}{5}
\usepackage[german]{babel}
\usepackage{mathtools}
\usepackage{tikz}
\usepackage{pgf}
\usepackage{csquotes}
\AtBeginDocument{
\renewcommand{\maketitle}{}
}
\PassOptionsToPackage{a4paper,margin = 2.5cm}{geometry}
\usepackage{geometry}
\newcommand{\bcenter}{\begin{center}}
\newcommand{\ecenter}{\end{center}}
\renewcommand{\contentsname}{Inhalt}
\usepackage{blindtext}
\usepackage[backend=biber, style = apa]{biblatex}
\addbibresource{Literatur.bib}
\ifLuaTeX
  \usepackage{selnolig}  % disable illegal ligatures
\fi
\usepackage[]{biblatex}
\addbibresource{Literatur.bib}
\usepackage{bookmark}
\IfFileExists{xurl.sty}{\usepackage{xurl}}{} % add URL line breaks if available
\urlstyle{same}
\hypersetup{
  pdftitle={Einleitung},
  pdfauthor={Franz Andersch \& Niklas Münz},
  hidelinks,
  pdfcreator={LaTeX via pandoc}}

\title{Einleitung}
\author{Franz Andersch \& Niklas Münz}
\date{2024-08-26}

\begin{document}
\maketitle

Warum eine gerade Ebene manchmal die beste Möglichkeit ist, um Daten in
zwei klassen aufzuteilen? Darum soll es in dieser Arbeit gehen. Wir
wollen hier die Methode der Support Vector Machines, die mit dieser
Annahme arbeitet, erläutern und gleichzeitig auf den Prüfstand stellen.
Denn auch wenn die Grundidee, vielleicht etwas zu simpel klingt, ist
ihre Leistung was vor allem binäre Klassifikationprobleme angeht
unumstritten. Die Grundlage für eine optimal seperierende Hyperebene
wurde bereits 1964 von Alexej Chervonenkis und Vladimir Vapnik gelegt
und die Methode wurde anschließend von mehrern Autoren stetig erweitert
\parencite{vapnikEstimationDependencesBased2006}. Zuerst wollen wir die
Funktionsweise der SVM näher beleuchten. Dazu wird zuerst gezeigt wie
die Konstruktion dieser seperiernden Hypereben in dem Fall funktioniert,
in dem die Daten tasächlich linear trennbar sind. Danach soll gezeigt
werden wie eine Soft Margin Klassifier funktioniert, der auch
funktioniert bei nicht perfekt linear trennbaren Daten und zuletzt soll
es um die Anwendung von Kernels gehen, die es dann auch ermöglichen,
nicht lineare Entscheidungsgrenzen herzustellen. im Kapitel 3 wollen wir
dann die Vor und Nachteile der Methode uns anschauen. Anhand dieser
Informationen konnten wir uns dann bereits einige Gedanken zur Leistung
der SVM in unserem Datenexperiment machen. Wie wir dieses Experiment
aufbauen soll im nachfolgenden Kapitel beschrieben werden. Wir
generieren uns zum anschließenden Vergleich neun verschiedene
Datenszenarien, welche in ihrer Dimensionalitäten und Kompleixität der
Entscheidungsgrenze variieren. Für den Vergleich wollen wir dann
Klassifikationsleistung von SVM uns anderen Klassifikationsmethoden
vegleichen. Im Teil Hypothesen haben wir aufgrund von Literatur, die
bereits ähnliche Versuche durchgeführt haben, Vermutungen aufgestellt,
wie die einzelnen Classifier im Vergleich abschneiden werden. Im
Ergebnis-Teil wollen wir dann das Experiment durchführen und mithilfe
von verschiedenen Maßzahlen die Leistung evaluieren. Ebenfalls wollen
wir üperprüfen, ob die Hypothesen die wir zuvor aufgestellt haben sich
bewahrheiten. Im letzen Teil soll ein Fazit gezogen werden, in dem Wir
die Ergebnisse noch einmal diskutieren und Defitite in unserer Arbeit
aufarbeiten.

\printbibliography

\end{document}
