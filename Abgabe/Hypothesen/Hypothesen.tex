% Options for packages loaded elsewhere
\PassOptionsToPackage{unicode}{hyperref}
\PassOptionsToPackage{hyphens}{url}
%
\documentclass[
]{article}
\usepackage{amsmath,amssymb}
\usepackage{iftex}
\ifPDFTeX
  \usepackage[T1]{fontenc}
  \usepackage[utf8]{inputenc}
  \usepackage{textcomp} % provide euro and other symbols
\else % if luatex or xetex
  \usepackage{unicode-math} % this also loads fontspec
  \defaultfontfeatures{Scale=MatchLowercase}
  \defaultfontfeatures[\rmfamily]{Ligatures=TeX,Scale=1}
\fi
\usepackage{lmodern}
\ifPDFTeX\else
  % xetex/luatex font selection
\fi
% Use upquote if available, for straight quotes in verbatim environments
\IfFileExists{upquote.sty}{\usepackage{upquote}}{}
\IfFileExists{microtype.sty}{% use microtype if available
  \usepackage[]{microtype}
  \UseMicrotypeSet[protrusion]{basicmath} % disable protrusion for tt fonts
}{}
\makeatletter
\@ifundefined{KOMAClassName}{% if non-KOMA class
  \IfFileExists{parskip.sty}{%
    \usepackage{parskip}
  }{% else
    \setlength{\parindent}{0pt}
    \setlength{\parskip}{6pt plus 2pt minus 1pt}}
}{% if KOMA class
  \KOMAoptions{parskip=half}}
\makeatother
\usepackage{xcolor}
\usepackage[margin=1in]{geometry}
\usepackage{graphicx}
\makeatletter
\def\maxwidth{\ifdim\Gin@nat@width>\linewidth\linewidth\else\Gin@nat@width\fi}
\def\maxheight{\ifdim\Gin@nat@height>\textheight\textheight\else\Gin@nat@height\fi}
\makeatother
% Scale images if necessary, so that they will not overflow the page
% margins by default, and it is still possible to overwrite the defaults
% using explicit options in \includegraphics[width, height, ...]{}
\setkeys{Gin}{width=\maxwidth,height=\maxheight,keepaspectratio}
% Set default figure placement to htbp
\makeatletter
\def\fps@figure{htbp}
\makeatother
\setlength{\emergencystretch}{3em} % prevent overfull lines
\providecommand{\tightlist}{%
  \setlength{\itemsep}{0pt}\setlength{\parskip}{0pt}}
\setcounter{secnumdepth}{5}
\usepackage[german]{babel}
\usepackage{mathtools}
\usepackage{tikz}
\usepackage{pgf}
\usepackage{csquotes}
\AtBeginDocument{
\renewcommand{\maketitle}{}
}
\PassOptionsToPackage{a4paper,margin = 2.5cm}{geometry}
\usepackage{geometry}
\usepackage{float}
\usepackage{enumitem}
\newcommand{\bcenter}{\begin{center}}
\newcommand{\ecenter}{\end{center}}
\renewcommand{\contentsname}{Inhalt}
\usepackage{blindtext}
\usepackage[backend=biber, style = apa]{biblatex}
\addbibresource{Literatur.bib}
\ifLuaTeX
  \usepackage{selnolig}  % disable illegal ligatures
\fi
\usepackage[]{biblatex}
\addbibresource{Literatur.bib}
\usepackage{bookmark}
\IfFileExists{xurl.sty}{\usepackage{xurl}}{} % add URL line breaks if available
\urlstyle{same}
\hypersetup{
  pdftitle={Hypothesen},
  pdfauthor={Franz Andersch \& Niklas Münz},
  hidelinks,
  pdfcreator={LaTeX via pandoc}}

\title{Hypothesen}
\author{Franz Andersch \& Niklas Münz}
\date{2024-08-22}

\begin{document}
\maketitle

Im Folgenden werden Studien hinzugezogen, um eine Einschätzung der
Performance in den verschiedenen Szenarien vorzunehmen und Hypothesen
abzuleiten. Vorab ist zu erwähnen, dass die Evaluation von
Klassifikationsmethoden anhand synthetischer Datensätze in der Literatur
begrenzt ist. Da für diese Arbeit die Form der Entscheidungsgrenze
entscheidend ist, werden dennoch auschließlich Arbeiten mit
synthetischen Datensätzen zu Rate gezogen.\newline Aufgrund dessen, dass
der Datengenerierende Prozess hier so ausgearbeitet wurde, dass er mit
den Annahmen der SVMs arbeitet, erwarten wir zuerst einmal eine bessere
Performance der SVM Classifier im Vergleich zu den anderen Methoden.

\begin{minipage}{0.9\linewidth}
\begin{itemize}[leftmargin=0.1\linewidth]
\item[\textbf{H1:}] Die SVM Classifier Performen über alle Datensituationen im Durchschnitt besser als die anderen Classifier
\end{itemize}
\end{minipage}

Des Weiteren wurden in den einzelnen Kategorien des daten generierenden
Prozesses die Entscheidungsgrenzen speziell auf verschiedene Kernels der
SVMs zugeschnitten. Daher sollten SVMs mit linearem Kernel im Setting
mit linearer Entscheidungsgrenze mindestens so gut oder besser als die
restlichen Classifier performen. Gleiches gilt für SVMs mit polynomialen
Kernel im Setting mit einer quadratischen Entscheidungsgrenze und
radiale Kernel bei einer Hypershäre als Entscheidungsgrenze.

\begin{minipage}{0.9\linewidth}
\begin{itemize}[leftmargin=0.1\linewidth]
\item[\textbf{H2:}] Die SVM Classifier mit dem Kernel, der für die jeweiligen DGP zugeschnitten ist sollten mindestens genauso gut oder besser Performen als die restlichen Classifier
\end{itemize}
\end{minipage}

Es konnte weiterhin gezeigt werden, dass in einem Szenario, indem
erheblich mehr Beobachtungen als Dimensionen und eine lineare
Entscheidungsgrenze vorliegen (S1), deutliche Unterschiede zwischen SVM,
k-NN und logistischer Regression bei der Diskriminationsfähigkeit
auftreten
\parencite{entezari-malekiComparisonClassificationMethods2009}. k-NN und
lineare SVM zeigen AUC-Werte nahe 1 auf, was für eine nahezu perfekte
Differenzierung der Klassen spricht. Die logistische Regression hingegen
hat einen Wert knapp über 0.5, was nur etwas besser als eine
Zufallsauswahl ist. Darüber hinaus ist festzustellen, dass die
Unterschiede deutlicher werden, je höher die Anzahl an Beobachtungen
ist.\newline Für den Fall einer radialen Entscheidungsgrenze (S3) sind
die Ergebnisse ähnlich. So erreicht in diesem Beispiel eine SVM mit
radialem Kernel im Vergleich zu einer logistischen Regression eine um
34\% höhere Genauigkeit
\parencite{faveroClassificationPerformanceEvaluation2022}.

Die Szenarien S4 bis S6 finden in der Literatur kaum Beachtung, weshalb
hier keine Studien herangezogen werden können. Liegt jedoch ein Szenario
vor, indem die Anzahl der Dimensionen erheblich größer ist, als die
Anzahl der Beobachtungen, mit einer linearen Entscheidungsgrenze (S7),
sind die Ergebnisse differenzierter zu betrachten. So schneidet die SVM
mit polynomialem Kern am besten unter den genannten Algorithmen ab,
jedoch die lineare SVM am schlechtesten (als Kriterium wurde die
mittlere Performance über 100 Datensätze evaluiert)
\parencite{scholzComparisonClassificationMethods2021}. Während k-NN auch
in diesem Szenario eine gute Performance hat, schneiden logistische
Regression und SVM mit radialem Kern mittelmäßig ab. Hierbei ist wichtig
zu erwähnen, dass in der Studie keine Ergebnisse über die genaue
Performance präsentiert wurden, sondern lediglich die Ränge der 25
behandelten Klassifikationsmethoden. Somit können nur eingeschränkte
Schlussfolgerungen gezogen werden.

Basierend auf den Ergebnissen der genannten Studien können folgende
Schlussfolgerungen gezogen werden.

\begin{minipage}{0.9\linewidth}
\begin{itemize}[leftmargin=0.1\linewidth]
\item[\textbf{H3:}] In niedrigdimensionalen Szenarien performen k-NN und SVM´s besser als eine logistische Regression.
\end{itemize}
\end{minipage}

Jedoch ist zu vermuten, dass die Wahl des Kerns bei SVM´s einen großen
Einfluss auf die Performance hat. So ist, basierend auf den
mathematischen Grundlagen, anzunehmen, dass die SVM mit dem jeweils
passenden Kern zu der vorliegenden Datensituation am besten performt. Da
die SVM mit polynomialem und radialem Kern weitaus flexibler sind,
werden diese voraussichtlich insgesamt betrachtet besser abschneiden als
die SVM mit linearem Kern.\newline

In hochdimensionalen Szenarien zeigt vermutlich die SVM mit polynomialem
oder radialem Kern eine gute Performance, unabhängig von der Form der
Entscheidungsgrenze, während die lineare SVM voraussichtlich weniger gut
abschneiden wird. Es scheint so, dass auch k-NN und logistische
Regression in hochdimensionalen Szenarien zumindest mittelmäßig
abschneiden. Es ist aber auch bekannt, dass gerade die k-NN Methode in
hochdimensionalen Settings schlechter performt
\parencite{jamesIntroductionStatisticalLearning2021}. Hier ist jedoch zu
beachten, dass nur eine lineare Entscheidungsgrenze betrachtet wurde und
in den Szenarien S8 und S9 andere Ergebnisse möglich sind. Wir schließen
Final daraus:

\begin{minipage}{0.9\linewidth}
\begin{itemize}[leftmargin=0.1\linewidth]
\item[\textbf{H4:}] In hochdimensionalen Settings performen v.a. SVMs mit radialen und polynomialen Kernel besser als die anderen Klassifikationsmethoden
\end{itemize}
\end{minipage}

\printbibliography

\end{document}
