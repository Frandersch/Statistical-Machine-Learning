% Options for packages loaded elsewhere
\PassOptionsToPackage{unicode}{hyperref}
\PassOptionsToPackage{hyphens}{url}
%
\documentclass[
]{article}
\usepackage{amsmath,amssymb}
\usepackage{iftex}
\ifPDFTeX
  \usepackage[T1]{fontenc}
  \usepackage[utf8]{inputenc}
  \usepackage{textcomp} % provide euro and other symbols
\else % if luatex or xetex
  \usepackage{unicode-math} % this also loads fontspec
  \defaultfontfeatures{Scale=MatchLowercase}
  \defaultfontfeatures[\rmfamily]{Ligatures=TeX,Scale=1}
\fi
\usepackage{lmodern}
\ifPDFTeX\else
  % xetex/luatex font selection
\fi
% Use upquote if available, for straight quotes in verbatim environments
\IfFileExists{upquote.sty}{\usepackage{upquote}}{}
\IfFileExists{microtype.sty}{% use microtype if available
  \usepackage[]{microtype}
  \UseMicrotypeSet[protrusion]{basicmath} % disable protrusion for tt fonts
}{}
\makeatletter
\@ifundefined{KOMAClassName}{% if non-KOMA class
  \IfFileExists{parskip.sty}{%
    \usepackage{parskip}
  }{% else
    \setlength{\parindent}{0pt}
    \setlength{\parskip}{6pt plus 2pt minus 1pt}}
}{% if KOMA class
  \KOMAoptions{parskip=half}}
\makeatother
\usepackage{xcolor}
\usepackage[margin=1in]{geometry}
\usepackage{graphicx}
\makeatletter
\def\maxwidth{\ifdim\Gin@nat@width>\linewidth\linewidth\else\Gin@nat@width\fi}
\def\maxheight{\ifdim\Gin@nat@height>\textheight\textheight\else\Gin@nat@height\fi}
\makeatother
% Scale images if necessary, so that they will not overflow the page
% margins by default, and it is still possible to overwrite the defaults
% using explicit options in \includegraphics[width, height, ...]{}
\setkeys{Gin}{width=\maxwidth,height=\maxheight,keepaspectratio}
% Set default figure placement to htbp
\makeatletter
\def\fps@figure{htbp}
\makeatother
\setlength{\emergencystretch}{3em} % prevent overfull lines
\providecommand{\tightlist}{%
  \setlength{\itemsep}{0pt}\setlength{\parskip}{0pt}}
\setcounter{secnumdepth}{5}
\usepackage[german]{babel}
\usepackage{mathtools}
\usepackage{tikz}
\usepackage{pgf}
\usepackage{csquotes}
\AtBeginDocument{
\renewcommand{\maketitle}{}
}
\PassOptionsToPackage{a4paper,margin = 2.5cm}{geometry}
\usepackage{geometry}
\newcommand{\bcenter}{\begin{center}}
\newcommand{\ecenter}{\end{center}}
\renewcommand{\contentsname}{Inhalt}
\usepackage{blindtext}
\usepackage[backend=biber, style = apa]{biblatex}
\addbibresource{Literatur.bib}
\ifLuaTeX
  \usepackage{selnolig}  % disable illegal ligatures
\fi
\usepackage[]{biblatex}
\addbibresource{Literatur.bib}
\usepackage{bookmark}
\IfFileExists{xurl.sty}{\usepackage{xurl}}{} % add URL line breaks if available
\urlstyle{same}
\hypersetup{
  pdftitle={Pros und Cons},
  pdfauthor={Franz Andersch \& Niklas Münz},
  hidelinks,
  pdfcreator={LaTeX via pandoc}}

\title{Pros und Cons}
\author{Franz Andersch \& Niklas Münz}
\date{2024-08-29}

\begin{document}
\maketitle

Support Vector Machines haben ein hohes Ansehen unter den Machine
Learning Algorithmen, da sie einige Vorteile mit sich bringen. Aufgrund
der Idee eines Soft Margin und des Kernel-Tricks ist die Methode sehr
flexibel und kann für spezielle Anwendungsbereiche angepasst werden
\parencite{bennettSupportVectorMachines2000}. Dazu sind die Ergebnisse
stabil und reproduzierbar, was sie von anderen Methoden wie
beispielsweise neuronale Netze abhebt. Auch die Anwendung ist
vergleichsweise einfach, da es eine überschaubare Anzahl an Parametern
gibt (wie beispielsweise bei der SVM mit radialem Kern nur der gamma-
und cost-Parameter festzulegen ist).\newline Die Möglichkeiten der
Nutzung verschiedener Kernel sind \textit{SVM} überaus vielseitig. Die
Auswahl des Kerns ermöglicht es, äußerst flexible Entscheidungsgrenzen
zu formen \parencite{kuhnAppliedPredictiveModeling2013}. Dadurch können
SVM an verschiedene Datensituationen angepasst werden.\newline Ein
weiterer Vorteil ist, dass die Methode weitgehend robust gegenüber
Overfitting ist \parencite{kuhnAppliedPredictiveModeling2013}. Dafür
verantwortlich ist der Cost-Parameter, anhand dessen der Fit an die
Daten kontrolliert werden kann. Jedoch birgt dies auch Probleme
(Erläuterungen im folgenden Abschnitt).\newline Diese Vorteile
resultieren in einer allgemein häufigen Nutzung von SVM in der
Wissenschaft. Sie haben folglich bewiesen, dass sie für verschiedenste
Aufgaben gut funktionieren
\parencite{kuhnAppliedPredictiveModeling2013}.

Trotz der vielfachen Nutzung von \textit{SVM}, bringen sie auch
Nachteile mit sich. Das wohl größte Problem liegt in der Modellselektion
\parencite{bennettSupportVectorMachines2000}. Wie bereits im vorherigen
Abschnitt erwähnt, ist die Auswahl der Parameter von hoher Bedeutung bei
der Performance und dem Fit an die Daten. So kontrollieren die
Kernel-spezifischen Parameter und der Cost-Parameter einerseits die
Komplexität und andererseits den Fit an die Daten
\parencite{kuhnAppliedPredictiveModeling2013}. Dabei kann die Wahl der
Parameter sowohl zu einem Underfit als auch zu einem Overfit führen.
Jedoch haben nicht nur die Parameter einen Einfluss auf die Performance
sondern bereits die Wahl des Kernels kann entscheidend sein
\parencite{burgesTutorialSupportVector1998}. Je nach Datensituation
können SVM mit verschiedenen Kernels äußerst unterschiedliche Ergebnisse
liefern. Dies zeigt die Sensibilität der Methode gegenüber der Wahl des
Kerns und der Parameterabstimmung.\newline Ein weiterer Nachteil ist,
dass die Methode weniger intuitiv und aufwendiger anzuwenden ist als
andere Algorithmen \parencite{bennettSupportVectorMachines2000}. So ist
es zum Beispiel schwer Informationen aus Support Vektoren zu ziehen und
es gibt keine Koeffizienten die interpretiert werden können.\newline
Zuletzt ist zu erwähnen, dass die Methode bei einer hohen Anzahl an
Beobachtungen besonders rechenintensiv ist. So konnte beispielsweise
gezeigt werden, dass insbesondere die SVM mit polynomialem und radialem
Kern eine hohe Rechenzeit aufweisen
\parencite{scholzComparisonClassificationMethods2021}. Dabei konnten
andere Methoden wie die \textit{logistische Regression} oder
\textit{k-nearest Neighbour} deutlich besser abschneiden. Dies liegt
daran, dass die Lösung des SVM-Optimierungsproblems die Behebung eines
quadratischen Programmierungsproblems erfordert. Da die Anzahl der zu
optimierenden Parameter mit der Anzahl der Daten quadratisch zunimmt,
führt dies zu einer hohen Rechenkomplexität
\parencite{kecmanSupportVectorMachines2005}.

\printbibliography

\end{document}
