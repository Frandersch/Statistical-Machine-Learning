% Options for packages loaded elsewhere
\PassOptionsToPackage{unicode}{hyperref}
\PassOptionsToPackage{hyphens}{url}
%
\documentclass[
]{article}
\usepackage{amsmath,amssymb}
\usepackage{iftex}
\ifPDFTeX
  \usepackage[T1]{fontenc}
  \usepackage[utf8]{inputenc}
  \usepackage{textcomp} % provide euro and other symbols
\else % if luatex or xetex
  \usepackage{unicode-math} % this also loads fontspec
  \defaultfontfeatures{Scale=MatchLowercase}
  \defaultfontfeatures[\rmfamily]{Ligatures=TeX,Scale=1}
\fi
\usepackage{lmodern}
\ifPDFTeX\else
  % xetex/luatex font selection
\fi
% Use upquote if available, for straight quotes in verbatim environments
\IfFileExists{upquote.sty}{\usepackage{upquote}}{}
\IfFileExists{microtype.sty}{% use microtype if available
  \usepackage[]{microtype}
  \UseMicrotypeSet[protrusion]{basicmath} % disable protrusion for tt fonts
}{}
\makeatletter
\@ifundefined{KOMAClassName}{% if non-KOMA class
  \IfFileExists{parskip.sty}{%
    \usepackage{parskip}
  }{% else
    \setlength{\parindent}{0pt}
    \setlength{\parskip}{6pt plus 2pt minus 1pt}}
}{% if KOMA class
  \KOMAoptions{parskip=half}}
\makeatother
\usepackage{xcolor}
\usepackage[margin=1in]{geometry}
\usepackage{graphicx}
\makeatletter
\def\maxwidth{\ifdim\Gin@nat@width>\linewidth\linewidth\else\Gin@nat@width\fi}
\def\maxheight{\ifdim\Gin@nat@height>\textheight\textheight\else\Gin@nat@height\fi}
\makeatother
% Scale images if necessary, so that they will not overflow the page
% margins by default, and it is still possible to overwrite the defaults
% using explicit options in \includegraphics[width, height, ...]{}
\setkeys{Gin}{width=\maxwidth,height=\maxheight,keepaspectratio}
% Set default figure placement to htbp
\makeatletter
\def\fps@figure{htbp}
\makeatother
\setlength{\emergencystretch}{3em} % prevent overfull lines
\providecommand{\tightlist}{%
  \setlength{\itemsep}{0pt}\setlength{\parskip}{0pt}}
\setcounter{secnumdepth}{5}
\usepackage[german]{babel}
\usepackage{mathtools}
\usepackage{tikz}
\usepackage{pgf}
\usepackage{csquotes}
\AtBeginDocument{
\renewcommand{\maketitle}{}
}
\PassOptionsToPackage{a4paper,margin = 2.5cm}{geometry}
\usepackage{geometry}
\newcommand{\bcenter}{\begin{center}}
\newcommand{\ecenter}{\end{center}}
\renewcommand{\contentsname}{Inhalt}
\usepackage{blindtext}
\usepackage[backend=biber, style = apa]{biblatex}
\addbibresource{Literatur.bib}
\ifLuaTeX
  \usepackage{selnolig}  % disable illegal ligatures
\fi
\usepackage[]{biblatex}
\addbibresource{Literatur.bib}
\IfFileExists{bookmark.sty}{\usepackage{bookmark}}{\usepackage{hyperref}}
\IfFileExists{xurl.sty}{\usepackage{xurl}}{} % add URL line breaks if available
\urlstyle{same}
\hypersetup{
  pdftitle={Pros und Cons},
  pdfauthor={Franz Andersch \& Niklas Münz},
  hidelinks,
  pdfcreator={LaTeX via pandoc}}

\title{Pros und Cons}
\author{Franz Andersch \& Niklas Münz}
\date{2024-08-15}

\begin{document}
\maketitle

\subsection{Pros}

\subsection{Cons}

\subsection{Hypothesen}

Die Performance von Support Vector Machines wird anhand verschiedener
Datenszenarien untersucht. Dabei werden Logistic Regression und
k-nearest Neighbour als Vergleichsalgorithmen hinzugezogen. Genauer
gesagt, wird in neun verschiedene Szenarien unterschieden, welche sich
durch zwei Unterteilungen ergeben: die Form der Entscheidungsgrenze
sowie das Verhältnis zwischen der Anzahl an Dimensionen (p) und der
Anzahl an Beobachtungen (n).Dabei werden ausschließlich binäre
Klassifikationen untersucht. Es ergibt sich folgende Aufteilung:

\begin{center}
\begin{tabular}{ |c|c|c|c| }
 \hline
  & linear & polynomial & radial \\
 \hline
 p << n & S1 & S2 & S3 \\
 \hline
 p = n & S4 & S5 & S6 \\
 \hline
 p >> n & S7 & S8 & S9 \\
 \hline
\end{tabular}
\end{center}

Im Folgenden werden Studien hinzugezogen, um eine Einschätzung der
Performance in den verschiedenen Szenarien vorzunehmen und Hypothesen
abzuleiten.\newline Es konnte gezeigt werden, dass in einem Szenario,
indem erheblich mehr Beobachtungen als Dimensionen und eine lineare
Entscheidungsgrenze vorliegen (S1), deutliche Unterschiede zwischen SVM,
k-NN und LogR bei der Diskriminationsfähigkeit auftreten
\parencite{entezari-malekiComparisonClassificationMethods2009}. k-NN und
lineare SVM zeigen AUC-Werte nahe 1 auf, was für eine nahezu perfekte
Differenzierung der Klassen spricht. LogR hingegen hat einen Wert knapp
über 0.5, was nur etwas besser als eine Zufallsauswahl ist. Darüber
hinaus ist festzustellen, dass die Unterschiede deutlicher werden, je
höher die Anzahl an Beobachtungen ist.\newline Für den Fall einer
radialen Entscheidungsgrenze (S3) sind die Ergebnisse ähnlich. So
erreicht in diesem Beispiel eine SVM mit radialem Kernel im Vergleich zu
einer LogR eine um 34\% höhere Genauigkeit
\parencite{faveroClassificationPerformanceEvaluation2022}.

Liegt ein Szenario vor, indem die Anzahl der Dimensionen erheblich
größer ist, als die Anzahl der Beobachtungen, mit einer linearen
Entscheidungsgrenze (S7), sind die Ergebnisse differenzierter zu
betrachten. So schneidet die SVM mit polynomialem Kern am besten unter
den genannten Algorithmen ab, jedoch die lineare SVM am schlechtesten
(als Kriterium wurde die mittlere Performance über 100 Datensätze
evaluiert) \parencite{scholzComparisonClassificationMethods2021}.
Während k-NN auch in diesem Szenario eine gute Performance hat,
schneiden LogR und SVM mit radialem Kern mittelmäßig ab.

Basierend auf den Ergebnissen der genannten Studien können
Schlussfolgerungen gezogen werden. Es ist anzunehmen, dass in
niedrigdimensionalen Szenarien k-NN und SVM´s besser performen als LogR.
Jedoch ist zu vermuten, dass die Wahl des Kerns bei SVM´s einen großen
Einfluss auf die Performance hat.\newline In hochdimensionalen Szenarien
zeigt vermutlich die SVM mit polynomialem oder radialem Kern eine gute
Performance, unabhängig von der Form der Entscheidungsgrenze, während
die lineare SVM voraussichtlich weniger gut abschneiden wird. Es scheint
so, dass auch k-NN und LogR in hochdimensionalen Szenarien zumindest
mittelmäßig abschneiden. Hier ist jedoch zu beachten, dass nur eine
lineare Entscheidungsgrenze betrachtet wurde und in den Szenarien S8 und
S9 andere Ergebnisse möglich sind.\newline \begin{align}
H1: Hypothese 1
\end{align}

\printbibliography

\end{document}
